

%src mongo lisence.


% https://www.techtarget.com/searchdatamanagement/definition/MongoDB
% jotain täältä. tämä tosin sanoo että on open source joka ei oo sinänsä totta
% sitaatti 28.5



\subsubsection{MongoDB yleisesti}


%MongoDB on mongoDB inc:in valmistama ja ylläpitämä, Server Side Public Lisenssillä toimiva NoSql (eng "non sql"{} tai "not only sql"{}) dokumentti pohjainen tietokanta hallinta ohjelma.

% this is from gpt with above prompt
%MongoDB is a NoSQL document-based database management program produced and maintained by MongoDB Inc.
%It is designed to handle large volumes of unstructured data and offers high performance, high availability, and easy scalability.
%MongoDB stores data in flexible, JSON-like documents, which means fields can vary from document to document and data structure can be changed over time.
%This database system is widely used for modern applications due to its ability to manage data of diverse types without requiring a fixed schema.

%mongo website and repo
% 5.6

%https://www.mongodb.com/docs/manual/core/document/
%json bson, document format
% 5.6

%https://www.techtarget.com/searchdatamanagement/definition/MongoDB
% performance and availability
% 5.6

%https://www.mongodb.com/resources/products/fundamentals/why-use-mongodb
%used alot
% 5.6


%sources are cited above
MongoDB on NoSQL(eng "non sql"{} tai "not only sql"{}) dokumenttipohjainen tietokannan hallintaohjelma, joka on MongoDB Inc:in valmistama ja ylläpitämä.
Se on suunniteltu käsittelemään suuria määriä strukturoimatonta dataa, ja se tarjoaa suuren suorituskyvyn, korkean käytettävyyden ja helpon skaalautuvuuden.
\medskip

MongoDB tallentaa tiedot joustaviin, JSON-tyyppisiin dokumentteihin, mikä tarkoittaa, että kentät voivat vaihdella dokumentista toiseen ja tietorakennetta voidaan muuttaa ajan myötä.
Tätä tietokantajärjestelmää käytetään laajalti nykyaikaisissa sovelluksissa, koska se pystyy hallitsemaan erityyppisiä tietoja ilman kiinteää skeemaa.
\medskip



\subsubsection{Skaalautuminen}


% https://www.mongodb.com/resources/products/capabilities/sharding
%5.6

Horisonttaalisella skaalautumisella tarkoitetaan sitä että lisätään koneita jotka pystyy jakaa datasetin tai loadin.
\medskip

% voi lisätä ylempään
mongodb skaalautuminen perustuu replikointi ja shardaus ominaisuuskiin
\medskip


%https://www.mongodb.com/resources/products/capabilities/replication
%5.6

replikointi antaa mahdollisuuden pitää monta db instannssia, jolloin jos pää db kaatuu voidaan käyttöönottaa kopio
kun primääri db saa kirjoitus operaation se jakaa sen secondäärien kanssa jolloin ne pysyvät ajan tasalla.
\medskip


%periaattees voi kirjoittaa verbatin suomeksi toi what is database sharding osa
%https://kinsta.com/blog/mongodb-sharding/
%5.6

% official 
% https://www.mongodb.com/resources/products/capabilities/sharding
%5.6


sharding antaa mahdollisuuden jakaa tietokanta moneen pienempään tietokantaan ja jakaa ne eri koneille. nämä pienemmät palat ovat shardeja.
%selitys miten toimii
\medskip

%koko sharn network hommassa on router ja config server ja shardit
%näistä selitys

shardissa on itse data ja se voi olla replikoituna

config server on lähde shardien metadatasta.

router. client interactaa tämän kanssa. router kommunikoi config serverin ja shardien välipllä että voi toimia
\bigskip




%täältä kuva
%https://www.mongodb.com/docs/manual/sharding/

% onko tämä kaavio vai kuva ja mitä väliä tai eroa

\includesvg{./src/public/oppar/mongosharding.svg}
Kuva\getImgCount{}. kaavio sharding mallista (mnongo source)
\medskip



\subsubsection{Nosql Dokumentti}

% selitä dokumentista. dokumenteista on jotenkin mainittu ylhäällä mutta tässä voi jatkojalostaa tai toistaa uusilla sanoissa 
% selitetäänkö miten se vertaa pöytään
% joku esimerkki kuva dokumentista


dokumentti struktuuri nosql kannassa on jollain tiedosto muodossa oleva dokumentti josta jonka voi tallentaa ilman skeemaa jne.
mongodb tallentaa dokumenttinsa bson muodossa
tietokannasta dokumentteja voi hakea niiden uniikkien tunnisteiden tai sisällön arvojen perusteella
\medskip


















% if we want to add stuff about finding document 
% though it seemed kinda difficult to find anything usefull 
%so probably not gonna write about this unless we need more content in here
    
%https://www.mongodb.com/docs/manual/indexes/
%https://www.mongodb.com/docs/v5.3/indexes/

% all sitated at 31.5

%incase i cannot read
%https://www.youtube.com/watch?v=nkSjhL40CTs
%https://www.youtube.com/watch?v=D14wWW9EEx8

% all sitated at 31.5

%https://www.mongodb.com/docs/manual/reference/method/cursor.explain/#mongodb-method-cursor.explain
% sitated at 5.6

