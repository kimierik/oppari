
\begin{tabular}{ | l | }

    %hack for the tiivistelmä line
    \multicolumn{1}{l}{
        \begin{minipage}{6cm}
            \hspace{1mm}
            
        \end{minipage}
        \begin{minipage}{4.25cm}
            \hspace{1mm}
        \end{minipage}
        \begin{minipage}[t][1.95cm][t]{3.62cm}
            \large
            \textbf{ Tiivistelmä }
        \end{minipage}
    }\\

    \hline
    \begin{minipage}[b]{6cm}
        Tekijä(t)
        \newline
        Kimi Malkamäki 
    \end{minipage}%
    % 2x2
    \begin{minipage}{8.5cm}
        \begin{tabular}{ | l | c | }
            \begin{minipage}[t][1cm][t]{4.25cm}
                Julkaisun laji
                \newline
                Opinnäytetyö, AMK
            \end{minipage} & %
            %
            \begin{minipage}{3.62cm}
                Valmistusaika
                \newline
                2024
            \end{minipage} \\ \hline%
            %
            \begin{minipage}[t][1cm][t]{4.25cm}
                Sivumäärä
                \newline 
                31+50
            \end{minipage}
            &  \\ \hline
        \end{tabular}
    \end{minipage}%
    % end of 2x2
      \\ \hline

    \begin{minipage}[t][2cm][t]{8cm}
    Työn nimi 
        \newline 
    ammatissa kasvaminen 
        \newline 
    oppimispäiväkirja  
    \end{minipage}\\ \hline

    \begin{minipage}[t][1.5cm][t]{10cm}
    Tutkinto ja koulutusala.\newline  Tieto- ja viestintätekniikka, insinööri AMK  

    \end{minipage}\\ \hline

    \begin{minipage}[t][7.5cm][t]{14.5cm}
    Tiivistelmä: \medskip 
    
    Opinnäitettyön tavoitteena oli seurata ammattimaista kehittymistä 13 viikon seurantajakson aikana, 
        jolloin kirjoitettiin oppimispäiväkirja. Päiväkirjaan on kirjattu haasteet ja ongelmat, mihin törmäsin harjoittelun aikana. \medskip

    Työtehtäviini seurantajakson aikana kuului 
        ominaisuuksien suunnittelu ja toteutus web alustaan ja sovelluksen ohjelmavirheiden korjaukset.
        Ominausuuksien toteuttamisessa ja sen prosessin dokumentoimisessa oppimispäiväkirjaan antaa hyvän silmäyksen alan ammattilaisen arkipäiväiseen työhön 
        ja haasteisiin mitkä ilmeentyvät päivittäin. \medskip

    Olen selvästi parantanut sosiaalisia ja teknisiä taitojani tämän kokemuksen myötä.
        Saamastani käytännön työkokemuksesta on epäilemättä hyötyä tulevalla urallani.
        Päiväkirjan avulla olen voinut pohtia edistymistäni ja tunnistaa alueita, joilla voin edelleen kehittyä. 

    \end{minipage}\\ \hline

    \begin{minipage}[t][2cm][t]{14cm}
    Avain sanat \medskip

    MeteorJS, ReactJs, MongoDB, Web suunnittelu, Responsiivinen suunnittelu
    \end{minipage}\\ \hline

\end{tabular}


\newpage

% ------------------------------ SECOND PAGE OF ABSTRACT ------------------------------ %

\begin{tabular}{ | l | }

    %hack for the tiivistelmä line
    \multicolumn{1}{l}{
        \begin{minipage}{6cm}
            \hspace{1mm}
            
        \end{minipage}
        \begin{minipage}{4.25cm}
            \hspace{1mm}
        \end{minipage}
        \begin{minipage}[t][1.95cm][t]{3.62cm}
            \large
            \textbf{ Abstract }
        \end{minipage}
    }\\

    \hline
    \begin{minipage}[b]{6cm}
        Author(s)
        \newline
        Kimi Malkamäki 
    \end{minipage}%
    % 2x2
    \begin{minipage}{8.5cm}
        \begin{tabular}{ | l | c | }
            \begin{minipage}[t][1cm][t]{4.25cm}
                Type of Publication 
                \newline
                Thesis, UAS?
            \end{minipage} & %
            %
            \begin{minipage}{3.62cm}
                Published
                \newline
                2024
            \end{minipage} \\ \hline%
            %
            \begin{minipage}[t][1cm][t]{4.25cm}
                Number of Pages
                \newline 
                31+50
            \end{minipage}
            &  \\ \hline
        \end{tabular}
    \end{minipage}%
    % end of 2x2
      \\ \hline

    \begin{minipage}[t][2cm][t]{8cm}
    Title of Publication
        \newline 
    ammatissa kasvaminen 
        \newline 
    oppimispäiväkirja  
    \end{minipage}\\ \hline

    \begin{minipage}[t][1.5cm][t]{10cm}
    Degree, Field of Stydy.\newline  Bachelof of information and communication Technologies

    \end{minipage}\\ \hline

    \begin{minipage}[t][7.5cm][t]{14.5cm}
    Abstract: \medskip 

    The objective of the study was to observe and document my professional development over a 13-week follow-up period.
        This was achieved by maintaining a learning diary, 
        in which i recorded the challenges and problems encountered during their training. \medskip

    The follow-up period entailed the design and implementation of new features for the web platform,
        as well as the resolution of technical issues within the application.
        The implementation of features and the subsequent documentation of the process in a learning diary provides insight into the daily work of a professional in the field and the challenges that they encounter on a daily basis. \medskip

    % not evident
    It is clear that this experiance has notably improved my social and technical abilities. 
        The practical work experience I have gained will prove advantageous in my future career.
        The diary has given me the chance to reflect on my progress and identify areas where i require further improvement.

    \end{minipage}\\ \hline

    \begin{minipage}[t][2cm][t]{14cm}
        Keywords
        \medskip

        MeteorJS, ReactJs, MongoDB, Web design, Responsive design 
    \end{minipage}\\ \hline

\end{tabular}
