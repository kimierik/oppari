

\subsection{Alkuperäinen ominaisuus ja parannuskohteet}





Sovelluksen "Progressio"{} sivulla on ominaisuus, 
joka antaa mahdollisuuden käyttäjälle seurata ja tallettaa hänelle annettua "mitattavaa"{} arvoa.
Kuvassa \nextImageCount {} progressiosivu, 
ja siitä käyttäjä pystyy valitsemaan, miltä päivältä arvo on otettu ja mikä itse arvo on.
\medskip

\bigskip
\includegraphics[width =15cm]{src/public/progressiosingle.png}\\
Kuva \getImgCount {}. Progressio sivun käyttöliittymä 
\medskip



Kuvassa \theimgCounter {} mitattavan arvon suure on paino. 
% 
Tällä hetkellä käyttäjälle voi määrittää vain yhden mitattavan arvon, jota seurata.
Sain tehtäväksi laajentaa ominaisuutta siten, että käyttäjälle voi määrittää useamman mitattavan arvon.
\medskip


Käyttäjän mitattavan arvon suure säilytetään käyttäjän profiilissa merkkijonona. 
Yksittäinen "measurableThing"{} ominaisuus ei ole riittävä useamman mitattavan arvon seurantaan, joten se pitäisi vaihtaa.
NoSql dokumenttitietokannan avusta "measurableThing"{} merkkijonon voi vaihtaa "measurableThings"{} merkkijonolistaan.
\medskip






\subsection{Ominaisuuden totetus}


\subsubsection{Käyttäjäprofiili skeeman migraatio}



Käyttäjän mitattavan arvon vaihtaminen listaan vaatii käyttäjän skeeman muutoksen.
MongoDB:n NoSql dokumentti ei määritä tiukkaa skeemaa, joten käyttäjien skeemat voivat olla eriävät.
Skeeman muutos vaatii jokaisen käyttäjän profiilin muokkaamista haluttuun muotoon.
Käyttäjäprofiilin skeeman muutos pitää tehdä kaikille aktiivisille käyttäjille, käyttäjän luontia pitää myös muokata siten, 
että uudet käyttäjät alkaisivat käyttämään uutta skeemaa.
\medskip

MongoDB antaa tallettaa mitä tahansa sen tietokantaan, joten projektissa on omat tyyppitarkastukset MongoDB rajapinnoissa, 
ettei virheelliset kirjoitusohjeet talleta mitään tietokantaan.
Nämä rajapinta tyyppitarkastukset pitää myös muokata sopimaan uuteen skeemaan.
\medskip

Migraatio tehdään kuvan \nextImageCount {} migraatiofunktiolla. 
Funktio käy läpi kaikki käyttäjät, joilla on olemassa measurableThing ominaisuus ja päivittää sen measurableThings listaan.
Projektissa on tarpeeksi vähän käyttäjiä, joten migraatio tehdään synkronisesti palvelimella. 
Jos projektissa olisi suuri määrä käyttäjiä tämä operaatio pysäyttäisi palvelimen migraation ajaksi, kun käyttäjäprofiileja päivitettäisiin.
Tällöin operaatio tehtäisiin huoltoaikana, jolloin sovellus ei olisi käytössä tai asynkronisesti, 
jolloin migraatio tapahtuisi pikkuhiljaa. 
\medskip

\bigskip
\includegraphics[width =15cm]{src/public/oppar/migrationfunction.png}\\
Kuva \getImgCount{}. Migraatio funktio
\medskip







\subsubsection{Käyttöliittymä}




Käyttöliittymä pitää muokata tukemaan ominaisuutta.
Käyttäjän luonti- ja päivityssivua pitää muokata siten, että käyttäjille voidaan antaa useita mitattavia arvoja.
Käyttäjän käyttöliittymää pitää muokata siten,
että käyttäjä voi vaihtaa mitä mitattavaa arvoa hän haluaa seurata tai mihin mitattavaan arvoon hän haluaa tallentaa lukemia.
\medskip

Kuvassa \nextImageCount{} on käyttäjäpuolen käyttöliittymä muutoksien jälkeen.
Käyttöliittymässä käyttäjällä on mahdollisuus valita mihin mitattavaan suureeseen hän haluaa tallettaa arvoja kuvan {\the\numexpr \theimgCounter + 2 } laskuvalikolla.
Sivun kaavioon on myös lisätty mahdollisuus seurata monta mitattavaa arvoa.
\medskip

\bigskip
\includegraphics[width = 15cm]{src/public/progressmulti.png}\\
Kuva \getImgCount {}. Käyttäjän käyttöliittymä, muutoksien jälkeen 
\medskip

\bigskip
\includegraphics{src/public/progressselect.png}\\
Kuva \getImgCount {}. käyttöliittymän laskkuvalikosta
\medskip



Kuvassa \nextImageCount {} on päivitetty käyttäjän luomis- ja päivityskäyttöliittymä.
Käyttöliittymään on lisätty nappi, josta voi lisätä, nimetä ja poistaa mitattavia suureita käyttäjälle.

\bigskip
\includegraphics[width = 10cm]{src/public/oppar/adminUserProfilePostcencoredroles.png}\\
Kuva \getImgCount {}. Käyttäjien luomiskäyttöliittymä 
\bigskip

Tekstikenttä ja poistonappi on sisällytetty kuvan {\the\numexpr \theimgCounter + 1 } React-komponenttiin "MeasurableThingElement". 
Tämä antaa mahdollisuuden lisätä uusia kenttiä, tekemällä uuden instanssin komponentista.

\includegraphics[width = 15cm]{src/public/oppar/measurableElementComponent.png}\\
Kuva \getImgCount {}. Mitattava elementti komponentti
\medskip

Komponentti on tilaton funktiokomponentti, joka saa mitattavan suurteen nimen ja poistofunktion input-parametreina.
Poistofunktiota kutsutaan, kun painetaan punaista X nappia ja se poistaa mitattavan suurteen käyttäjän profiilista ja elementin käyttöliittymästä.
\medskip












