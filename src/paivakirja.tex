\iffalse


Päiväkirjan .tex tiedosto
alla on testmode definition. sen päälle laittaminen kertoo sectionilla olevat sanamäärät ja sivun kokonais sanat

että testi moodi toimii kunnolla pitää olla '--shell-escape' optio millä ikinä kääntääkään tämän tiedoston

kaikki koodi screenshotit tehdään normaali sonokai väriteemalla

\fi

% Definet
\def\testmode{1}






\documentclass[11pt,a4paper,titlepage,oneside]{article}

\usepackage[most]{tcolorbox}
\usepackage{geometry}
\usepackage{hyperref} %links
\usepackage[document]{ragged2e} %floating text
\usepackage{helvet} %font
\usepackage{tocloft} % dots in table of contents sections
\usepackage{graphicx}


\usepackage{lipsum}
\usepackage{fancyhdr}

% hypthenationrules
\usepackage[T1]{fontenc}



\hypersetup{ colorlinks, citecolor=black, filecolor=black, linkcolor=black, urlcolor=black }
\urlstyle{same}



\geometry{ a4paper, left=3cm, right=2.5cm, top=3cm, bottom=2.5cm }



%custom reusable colorbox rules
\newtcolorbox{simplebox}{colback=white, sharp corners, boxrule=1pt }

\renewcommand{\contentsname}{Sisällys} % override Contents => Sisallys on table of contents

\renewcommand{\familydefault}{\sfdefault} % set font

\renewcommand{\cftsecleader}{\cftdotfill{\cftdotsep}} % dots for sections

\def\filename{src/paivakirja.tex}

% if i change this to be every week is one file then we need to change this to start reading rrom subcounts onwards
%or dont use the thing tsection command
% or rewrite the tsection command






\newcommand\wordcount{
    \immediate\write18{texcount -sub=section \filename{} -inc  | grep Section | sed -e 's/+.*//' | sed -n \thesection p > 'count.txt'}
(\input{count.txt}words)}


\newcommand{\inputf}[1]{\def\filename{#1}\input{#1}}


\newcommand\filewcount{
    \immediate\write18{texcount -1 -sum \filename{} > count.txt }
    \input{count.txt} words }


% jos on testi definetty niin tee 
\newcommand{\istest}[2]{\ifx\testmode\undefined #2 \else #1 \fi}


% testi section näyttää word countin siltä sectionilta jos ei testi niin sitten näytä vain normaali section
\newcommand{\tsection}[2]{\istest{#1{#2\\} \wordcount \\ \medskip }{#1*{#2}}}


\newcommand{\pagesection}[1]{\istest{\section{#1}  }{\section*{#1}}}
\newcommand{\pagesubsection}[1]{\istest{\subsection{#1}  }{\subsection*{#1}}}


% adds page with everything
\newcommand{\addPage}[2]{
    \def\filename{#1}
    \pagesection{#2}
    \istest{ \filewcount \medskip }{}
    \input{#1}
}



% OPPARI SPESIFIC

\newcommand{\addPageOp}[2]{
    \def\filename{#1}
    \pagesubsection{#2}
    \istest{ \filewcount \medskip }{}
    \input{#1}
}

%jos pitää päivittää viitattu sanaa myöhemmin vaikka lisätä . tai : tai jotain
\def\viitattu{viitattu}

%lab citation style
\newcommand{\labcite}[1]{\setcitestyle{aysep={},open={(},close={.)}}\citep{#1}{}}

%lab citation style
\newcommand{\labciteend}[1]{\setcitestyle{aysep={},open={(},close={).}}\citep{#1}{}}

\newcommand{\labimgcite}[1]{\setcitestyle{aysep={},open={(},close={)}}\citep{#1}{}}

\newcommand{\citemissing}{\textbf{( SRC ? )}}

%\newcommand{\hurl}[1]{\href{#1}{{\underline{\textcolor{blue}{#1}}}}}
\newcommand{\hurl}[1]{\textcolor{blue}{\url{#1}}}


% counter kuville
\newcounter{imgCounter}
\setcounter{imgCounter}{0}

\newcommand{\getImgCount}{\addtocounter{imgCounter}{1}\theimgCounter}

\newcommand{\nextImageCount}{\the\numexpr \theimgCounter + 1 }
\newcommand{\nextnextImageCount}{\the\numexpr \theimgCounter + 2 }
\newcommand{\prevImageCount}{\the\numexpr \theimgCounter - 1 }

% counter kaavioille
\newcounter{chartCounter}
\setcounter{chartCounter}{0}

\newcommand{\getChartCount}{
\addtocounter{chartCounter}{1}
\thechartCounter
}





%hyphenation rules

\usepackage[finnish]{babel}
















% RANDOM TODO
%
% fix indentation and sections
%   rn top section is the title of the page and should be something else i think
%   title should be title not section?
%
% read writing instructions, read text
% class komponentti => luokkakomponentti
% dependency => riippuvuudet/ riippuvuus
% onko funktio komponentti yhdyssana
% välilehti joku toinen sillä me ollaan samalla välilehdellä vaan siirrytään eri sivulle?
%   navigointi reittien välillä?


% sivu reitti välilehti 
% kai se on sivu



% NEXT MEET
% 4 weeks of diary
% fix first weeks grammar aswell





\iffalse
———————————No Ideas?——————————
⠀⣞⢽⢪⢣⢣⢣⢫⡺⡵⣝⡮⣗⢷⢽⢽⢽⣮⡷⡽⣜⣜⢮⢺⣜⢷⢽⢝⡽⣝
⠸⡸⠜⠕⠕⠁⢁⢇⢏⢽⢺⣪⡳⡝⣎⣏⢯⢞⡿⣟⣷⣳⢯⡷⣽⢽⢯⣳⣫⠇
⠀⠀⢀⢀⢄⢬⢪⡪⡎⣆⡈⠚⠜⠕⠇⠗⠝⢕⢯⢫⣞⣯⣿⣻⡽⣏⢗⣗⠏⠀
⠀⠪⡪⡪⣪⢪⢺⢸⢢⢓⢆⢤⢀⠀⠀⠀⠀⠈⢊⢞⡾⣿⡯⣏⢮⠷⠁⠀⠀
⠀⠀⠀⠈⠊⠆⡃⠕⢕⢇⢇⢇⢇⢇⢏⢎⢎⢆⢄⠀⢑⣽⣿⢝⠲⠉⠀⠀⠀⠀
⠀⠀⠀⠀⠀⡿⠂⠠⠀⡇⢇⠕⢈⣀⠀⠁⠡⠣⡣⡫⣂⣿⠯⢪⠰⠂⠀⠀⠀⠀
⠀⠀⠀⠀⡦⡙⡂⢀⢤⢣⠣⡈⣾⡃⠠⠄⠀⡄⢱⣌⣶⢏⢊⠂⠀⠀⠀⠀⠀⠀
⠀⠀⠀⠀⢝⡲⣜⡮⡏⢎⢌⢂⠙⠢⠐⢀⢘⢵⣽⣿⡿⠁⠁⠀⠀⠀⠀⠀⠀⠀
⠀⠀⠀⠀⠨⣺⡺⡕⡕⡱⡑⡆⡕⡅⡕⡜⡼⢽⡻⠏⠀⠀⠀⠀⠀⠀⠀⠀⠀⠀
⠀⠀⠀⠀⣼⣳⣫⣾⣵⣗⡵⡱⡡⢣⢑⢕⢜⢕⡝⠀⠀⠀⠀⠀⠀⠀⠀⠀⠀⠀
⠀⠀⠀⣴⣿⣾⣿⣿⣿⡿⡽⡑⢌⠪⡢⡣⣣⡟⠀⠀⠀⠀⠀⠀⠀⠀⠀⠀⠀⠀
⠀⠀⠀⡟⡾⣿⢿⢿⢵⣽⣾⣼⣘⢸⢸⣞⡟⠀⠀⠀⠀⠀⠀⠀⠀⠀⠀⠀⠀⠀
⠀⠀⠀⠀⠁⠇⠡⠩⡫⢿⣝⡻⡮⣒⢽⠋⠀⠀⠀⠀⠀⠀⠀⠀⠀⠀⠀⠀⠀⠀
—————————————————————————————
\fi


% search terms

% VIIKKO







\begin{document}     % ------------------------------ BEGIN DOCUMENT ------------------------------ %


\tsection{\section}{Päiväkirja}

%total word count of document
\istest{ \immediate\write18{texcount src/paivakirja.tex | grep 'Words in text' | sed -e 's/Words\ in\ text://' > 'count.txt'}
(\input{count.txt}total words) \\ }


\iffalse
päiväkirja on jaettu teemoihin, kyseiset teemat voivat kestää useamman viikon ja ne sisältää aihe kokonaisuuden.
- jaettu teemoihin 
- teemat voivat mennä useamman viikon
- rakenne on. suunnitelma - tekele - yhteenveto
- 13vk mitä tein teema muodossa \\medskip
\fi


% suurin piirtein mitä tein siellä.   olin enemmän kun 13vk starttaamolla 
% kertoo vaihtoon
Päiväkirja kertoo 13 viikon aikaisesta seurantajaksosta työharjoituksessa Starttaamo oy:ssä.
% tikettien ratkominen oikea sana/ en tiedä pitäiskö tähän tyyliin olla vähän enemmän tekstiä //gipity??
Työtehtäviini kuului sivujen ylläpito, testaus, käyttöliittymän muutokset, ominaisuuksien lisääminen ja tikettien käsittely.
\medskip

Päiväkirja on jaettu teemoihin, jotka voivat kestää useamman viikon. Teemassa käydään teemaan liittyvät asiat ja mitä niiden viikkojen aikana tehty työ.
\newpage









\addPage{./src/pv/kehitysympäristö_vk1.tex}{Kehitysympäristön pystytys projektin ehdoin (Viikko 1)}% --------------------------------------------------  VIIKKO 1 -------------------------------------------------- %   
\newpage


\addPage{./src/pv/responsiivinen_vk2.tex}{Responsiivisen käyttöliittymän tekeminen (Viikko 2)}% --------------------------------------------------  VIIKKO 2 -------------------------------------------------- %   
\newpage


\addPage{./src/pv/react_vk3.tex}{React (Viikko 3)}                                  % --------------------------------------------------  VIIKKO 3 -------------------------------------------------- %   
\newpage


\addPage{./src/pv/localisation_vk4.tex}{Lokalisaatio (Viikot 4-5)}                          % --------------------------------------------------  VIIKKO 4 - 5 -------------------------------------------------- %   
\newpage


\addPage{./src/pv/meteor_vk6.tex}{Meteor(Viikko 6)}                          % --------------------------------------------------  VIIKKO 6 -------------------------------------------------- %   
\newpage


% https://www.mongodb.com/nosql-explained/nosql-vs-sql#what-are-the-benefits-of-nosql-databases
\addPage{./src/pv/nosql_vk7.tex}{nosql (Viikko 7-8)}                          % --------------------------------------------------  VIIKKO 7-8 -------------------------------------------------- %   
\newpage


\addPage{./src/pv/feature_vk10.tex}{Uusi ominaisuus (Viikko 10-11)}                          % --------------------------------------------------  VIIKKO 10-11 -------------------------------------------------- %   
\newpage


\addPage{./src/pv/tba_vk12.tex}{jotain bullshit (Viikko 12)}                          % --------------------------------------------------  VIIKKO 12 -------------------------------------------------- %   
\newpage



\addPage{./src/pv/docker_vk13.tex}{docker ja siirrettävä kehitysympäristo (viikko 13)}                          % --------------------------------------------------  VIIKKO 13 -------------------------------------------------- %   
\newpage






\end{document}
