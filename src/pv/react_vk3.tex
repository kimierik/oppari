
Kolmannella viikolla olin Reactin kanssa töissä. Tein pieniä käyttöliittymän korjauksia ja yleistä ylläpitoa. Tutustuin miten projekti käyttää ReactJs kirjastoa ja miten käyttöliittymä on rakennettu.\medskip

Tuli myös ilmi, että uusien sivujen lisääminen toisi vaikeuksia mobiilikäyttöliittymään. Navigaatiopalkki mobiililaitteilla on nyt jo ahdas, ja uusien sivujen lisääminen tekisi siitä käyttö-kelvottoman.
Viikolla työnaiheena tuli myös testata navigaatiopalkkin siirtämistä sivun vasempaan reunaan, jossa sitä voisi rullata alas, jos se täyttyisi kokonaan.\medskip




\subsection*{ Reactin käyttö projektissa }


Starttaamon käyttöliittymä on rakennettu Reactjs kirjaston avulla. Hyödyntäen sen laajaa ekosysteemiä ja skaalautuvaa kehitysprosessia, käyttöliittymä voi mukautua startupin nopeaan kehitykseen ja yllättäviin muutoksiin nopeasti.
% gpt promt:the project is build ontop of an older code base so older components are class components and newer components are function components
Projekti on rakennettu olemassa olevan koodipohjan päälle, jossa vanhemmat komponentit ovat luokkakomponentteja, toisin kuin uudemmat lisäykset, jotka ovat funktiokomponentteja.
% gpt prompt: components that require state are using either the useState hook if it is a funktion component or the this.state property if it is a class component
Komponentit, jotka vaativat tilan käyttöä käyttävät joko useState hookkia funktiokomponentin tapauksessa tai this.state ominaisuutta luokkakomponentin tapauksessa. Projektissa ei ole erillistä tilanhallintaa.\medskip

% ei puhuta projektista mutta ei varmaan väliä
React-komponentit renderöi uudelleen, kun niiden tila muuttuu tai päivittyy. Mutta react komponentit eivät uudelleen renderöi itseään, jos tietokannassa oleva data muuttuu. 
Yleisesti tälläisessä tapauksessa pitäisi joko ajoittain tarkastaa datan tila tietokannasta tai palvelin lähettäisi tiedon käyttäjälle tietokannan päivityksestä esimerkiksi WebSockettia käyttäen.
Projektin käytössä oleva Meteor päivittää tarvittavat komponentit, kun se huomaa komponentin käyttämän datan muutoksen tietokannassa.\medskip

Komponenttien tyylitys on sekoitus CSS-tiedostoja ja styled-components kirjaston avulla tehtyjä CSS-sääntöjä.
Vanhempien luokkakomponenttien tyylitys on hoidettu CSS-tiedostojen avulla.
Uudemmat funktiokomponentit käyttävät styled-components kirjastoa tyylityksessä. Kirjasto antaa CSS-luokille uniikin nimen, tämä auttaa estämään virheitä joissa kahdella komponentilla on saman niminen CSS-luokka eri CSS-säännöillä.\medskip



\subsection*{ Navigaatiopalkki }

Navigaatiopalkin siirtämisellä haluttiin ratkaista ongelma mobiilikäyttöliittymän ahtaudesta. Jos navigaatiopalkki olisi sivun vasemmassa reunassa siihen voisi lisätä pystysuuntaisen rullauksen, jolloin kuvien kokoa ei tarvitsisi muuttaa.
\medskip



% atleast improve the image somehow crop it or something
\includegraphics{src/public/starttaamohomenavbar.png} \\
kuva starttaamon käyttäjäpuolen navigaatio palkista \medskip

Uudelleen käyttämisen ja helpomman ylläpidon vuoksi navigaatiopalkista tehtiin komponentti.
% uusiks jotenkin
Navigaatiopalkin ei tarvitse olla tilallinen komponentti, sillä se vain ohjaa käyttäjää välilehtien välillä eikä siinä ole mitään mitä pitäisi päivittää.
Itse navigoinnin hoitaa linkit toisille sivuille, ja koska ne ovat kovakoodattuja, komponentti ei tarvitse proppeja tekien siitä on itsenäisesti toimivan. 
\medskip


Styled-components kirjastolla työn tekeminen on melkein sama kuin tavallisella CSS:llä. Mutta toisin kuin tavallisessa CSS-määrittelyssä, styled-components tehdään JavaScript tiedostoon, jossa annetaan CSS-määrittely tekstinä kirjastolle "tagged template literal" JavaScript syntaksilla.
Tämä luo komponentin, jonka kaikki lapsikomponentit saa käyttöönsä annetut CSS-luokat. Luokat saavat uniikit tunnukset, joka estää ongelmia eri komponenttien saman nimisillä CSS-luokilla.
\medskip

Lisäksi komponentilla lisättiin mahdollisuus piiloutua ja tulla esille. Napista työpöytäkäyttöliittymällä tai pyyhkäisemällä mobiililaitteella, näin lisäten rajallista ruututilaa silloin kun navigaatiopalkkia ei tarvitse.\medskip




% kuva lopullisesta navigaatio palkista?





\subsection*{Viikon Yhteenveto}

% passiivi?
Viikon aikana opin miten projekti käytti ReactJs kirjastoa sen käyttöliittymän tekemiseen. 
Loin myös itsenäisesti toimivan komponentin navigaatiopalkille, jota voi helposti uudelleen käyttää.\medskip

Navigaatiopalkkia ei siirretty sivun vasempaan reunaan, sillä se olisi ollut liian sekava aktiivisille käyttäjille. 
Komponentti on silti olemassa, joten sen voi käyttöönottaa tulevaisuudessa jos sen tarve ilmenee.\medskip
