

Startupin nopean kehityksen ohella mobiili-käyttöystävällisyyttä ei ollut otettu huomioon, 
joten kehitysympäristön käynnistettyä sain aiheeksi muuttaa sovelluksen käyttöliittymän mobiiliystävälliseksi.
Projektiin ei haluttu lisätä erillistä mobiilisovellusta, vaan samaa nettisivua pitäisi pystyä käyttämään mobiili- ja työpöytälaitteilla, joten sivu oli tehtävä responsiiviseksi.\medskip





\subsection*{Responsiivinen käyttöliittymä}




% fine? i think
Responsiivinen verkkosuunnittelu on lähestymistapa verkkosivustojen rakentamiseen, jolla varmistetaan niiden optimaalinen ulkoasu ja toimivuus eri laitteilla ja näytön koossa.
Käyttämällä responsiivisia verkkosuunnittelua verkkosivustot voivat tarjota yhdenmukaisen käyttökokemuksen eri laitteilla.
\medskip


Web-käyttöliittymissä elementtien koot voivat olla absoluuttiset tai suhteelliset niiden ylempään elementtiin verrattuna. Absoluuttiset arvot tarkoittavat, että elementti olisi aina tietyn kokoinen pikseleissä. 
Tämä voi luoda tilanteita, jossa pienempikokoisemman näytöllä joku elementti on liian iso, tai ei sovi enää käyttöliittymään, joten absoluuttisia ei voi käyttää responsiivisessa käyttöliitymässä.
Relatiiviset elementtikoot eivät kuitenkaan ratkaise kaikkia ongelmia. Mobiili- ja työpöytälaitteiden kuvasuhteet eroavat. Käyttöliittymän pitää olla erilainen jos pituutta on enemmän, kun leveyttä. 
Ja, koska työpöytälaitteita käytetään hiirellä, joka antaa tarkan osoittimen käyttäjälle, toisin kuin mobiilissa, jota käytetään sormilla. 
Pitää ottaa huomioon linkkien, nappien ja kuvakkeiden koko, jotta itse sovelluksen käyttäminen olisi huoletonta.\medskip






CSS Media Query antaa mahdollisuuden CSS-säännöksen asettamiselle vain silloin kun selain- ja laiteympäristö vastaa määritettyjä sääntöjä. 
Säännökset voivat olla esimerkiksi "jos näytön koko on pienempi kuin X".
Media Queryt ovat tärkeä osa repsonsiivista suunittelua ja se antaa mahdollisuuden luoda erilaisia layouttteja riippuen käyttölaitteesta, sen näytön koosta tai kuvasuhteesta.
\medskip












\subsection*{Viikon Yhteenveto}


Responsiivinen web-suunnittelu on lähestymistapa nettisivujen kehittämiseen, jotka pystyvät mukautumaan käyttölaitteisiin. Sen osaaminen on tärkeä osa modernia web-kokemusta, ja se on oleellinen osata web-kehittäjänä.
\medskip


Responsiivinen web-suunnittelu ei ole pelkästään elementtien kokojen muokkaamista, vaan koko näkymän muuntamista.
Mobiililaitteiden eroavan kuvasuhteen ja pienemmän koon takia sivun layout pitää muuntaa sopivaksi elementtejä siirrellen siten, että käyttökokemus pysyy hyvänä. 

\newpage
