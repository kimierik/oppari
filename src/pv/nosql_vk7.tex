
mitä tehtiin viikolla?
miksi käytettiin 2 viikkoa tähän.
%admin kk translate roolitus
%8vk  tehtiin rooleja mutta se ei ole oikein mongo hommaa
%niin tässä ei oikein kerrota mittään siitä mitä tehdiin viikonaikana vaan kirjoitetaan suoraan tietokannoista i quess



\subsection*{nosql}
mikä on nosql kanta
toisin kun relatiiviset joilla on teitty row ja collumni määrä non sql voi olla esim. avain-arvo dockumentti graafi jne
sql schma on rigid toisin kun nosql jossa se voi vahdella

relatiivinen välttää usein kalliita join operaatioita
%selitä kirjakielellä ei puhekielellä
% huono esimerri muutenkin
esim jos olisi sql tietokanta olisi, pöytä asiakkaille jossa on asiakas id, ja pöytä ostoksille jossa olisi kanssa asiakas id
jos halutaan kaikki asiakkaan ostokset nimellä niin ekaksi pitää etsiä asiakas id asiakas pöydästä sitten asiakas idnkanssa etsiä ostokset ostos pöydästä.
% hmmmmm
toisin kun esim dokumentti tietokannassa asiakas tiedossa voi olla monta muuta juttua. esim natiiveja+* tyyppejä kuten string, int float tai array.


\subsection*{mongodb}

%suomenna
mongodb on proprietary? nosql tietokanta, joka tallentaaa sen tiedostot BSON muodossa dokumentteihin. BSON on binääri representaatio json formaatista .
mongodbseen ei tarvitse määrittää millainen schema datalla on
mongo myös tukee sharding eli voi jakaa databasen tiedon clusterille joka antaa mahdollisuuden vaakatsuuntaisselle skaalaavuuldelle


mongo projektissa\\
mongoDBtä käytetään käyttäjätietojen, treenitietojen, seurantatietojen jne kanssa
periaatteessa kaikkien muiden kanssa paitsi tiedostojen kanssa jotka on aws bucketissa
koska mongogodb ei kiinnosta missä schemassa data on. ennen mongo api kutsuja meillä on itse tyyppi varmistelu ettei mitään ylimääräistä mene tietokantaan vahingossakaan.



\subsection*{yhteenveto}
jotain tekstiä
