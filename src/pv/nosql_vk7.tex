

tietokannat niiden eri tyypit ja käytetty mongodb tietokanta ja miten projekti käyttää sitä.




\subsection*{Relaationainen tietokanta}

%https://www.ibm.com/topics/relational-databases

Relationaalinen tietokanta on tietokanta malli, joka organisoi tiedot riveihin ja sarakkeisiin, jotka muodostavat pöydän. 
Pöydän sarakkeet määrittävät sen skeeman ja rivit jokaisen tietueen.
Jokaisella rivillä on joku uniikki tunniste millä sen voi löytää. Tätä kutsutaan primääri avaimeksi.
\medskip

Ulkoinen avain (eng foreign key) on sarake tai sarakkeet, jotka viittaa toisen pöydän primääri avainta, muodostaen linkin pöytien välillä.
Ulkoisia avaimia voi näin käyttää taulukkojen ristivuuttaukseen, joka antaa mahdollisuuden yhdistää pöytiä, joilla on eri skeema, esim. asiakastiedot ja ostostiedot. 
\medskip


\medskip

\subsection*{Nosql tietokanta}


Nosql (not only sql tai non sql) on ei relationaalinen tietokanta.
Toisin kun relationaaliset tietokannat, ei relationaalinen tietokanta ei käytä tarkkaa skeemaa eikä välttämättä sarake - rivi mallia, eli datan skeema voi olla dynaaminen ja vaihteleva.
Ei relationaaliset tietokannat voivat käyttää erilaisia tiedon haku ja tallennus järjestelmiä, yleisimmät ovat graph, document ja key-value.
\medskip



Graafin tietokannan tiedot on organisoitu solmuihin (eng node) ja suhteisiin (eng relation).
Solmu on tietopiste, johon voi tallettaa dataa skeematta. 
%https://neo4j.com/docs/getting-started/appendix/graphdb-concepts/
Suhde on yhteys kahden solmun välillä. Suhteissa voi myös olla tallennettuna ominaisuuksia, joiden avulla voi etsiä solmuja,
esimerkiksi jos solmut:t esittävät ihmisiä, niissä voisi olla ominaisuudet, jotka kuvaavat heidän nimeänsä ja syntymäaikaa.
Suhteet olisivat yhteyksiä ihmisten välillä, ja ne voivat kuvata henkilöiden suhteita.
\medskip

Dokumentti pohjainen relationaalinen kanta, jota projekti käyttää, voi tallentaa datan eri tiedostomuodoissa esim. JSON tai YAML
Joka dokumentilla on uniikki tunniste, joka erottaa sen muista. 
Dokumentteja voi noutaa tietokannasta uniikin tunnisteen tai ominaisuuksien arvojen avulla.
\medskip

Avain-arvo (eng key-value) tietokanta on käytännössä hajautustaulu (eng hash table).
Jokaisella tietopisteellä on uniikki avain, joka viittaa johonkin arvoon. 
Avaimesta lasketaan arvon osoite hajautusfunktiota käyttäen, joten etsinnän nopeus ei riipu datan kokonaismäärästä.
\medskip




\subsection*{Mongodb}

Mongodb on nosql tietokanta, joka käyttää dokumenttimaista tietomallia, johon data on tallennettuna BSON muotoon, joka on binaari representaatio JSON formaatista.
\medskip

Projektissa mongodb:tä käytetään kaiken muun kuin tiedostojen tallentamiseen. Tiedostot projektissa on AWS S3 ämpärissä
Koska relatiivinen dokumenttimalli ei määritä tarkkaa skeemaa, mongodb antaisi mahdollisuuden tallentaa mitä tahansa dataa tietokantaa.
%suomenna wrapper... do we need to?
Turvallisuus syistä projektin käyttöliittymä ja palvelin koodissa ei kutsuta suoraan mongodb rapapinta funktioita vaan niille on tehty omat "wrapper"{} funktiot.
Funktiot katsovat ja varmistavat datan mitä yritetään kysyä tai tallentaa tietokantaan, ettei sieltä saa tai sinne voi tallentaa mitään ylimääräistä.
Näin voimme luoda lisä turvallisuutta tietojenhallinnassa.
\medskip



\subsection*{yhteenveto}

%Projekti ei vaadi käyttää relatiivistä tietokantaa mutta se tekee joistakin operaatioista helpompia, sillä skeema on dynaamisempi ja voimme tallentaa listoja dokumentteihin.

Projekti ei itsessään vaadi ei relationaalista tietokantaa vaan sen pystyisi yhtä hyvin toteuttamaan relationaalisella tietokannalla, 
mutta ei relationaalinen kanta auttaa joissakin operaatioissa.
Esimerkiksi seuraavan viikon projektissa. Sillä siellä mainittujen mitattavien arvojen lisäämisen ja etsimisen on helpompaa.
Relationaalisessa tietokannassa meillä olisi oma pöytä, jossa mitattavat arvot tallennettaisiin.
Jos haluttaisi tietää käyttäjän mitatut arvot nämä pöydät pitäisi yhdistää.
Toisin kun ei relationaalisessa, jossa voimme vain hakea käyttäjän tiedot ja siinä mukana on lista mitatuista arvoista.



