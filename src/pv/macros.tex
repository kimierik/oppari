

\newcommand\wordcount{
    \immediate\write18{texcount -sub=section \filename{} -inc  | grep Section | sed -e 's/+.*//' | sed -n \thesection p > 'count.txt'}
(\input{count.txt}words)}


\newcommand{\inputf}[1]{\def\filename{#1}\input{#1}}


\newcommand\filewcount{
    \immediate\write18{texcount -1 -sum \filename{} > count.txt }
    \input{count.txt} words }


% jos on testi definetty niin tee 
\newcommand{\istest}[2]{\ifx\testmode\undefined #2 \else #1 \fi}


% testi section näyttää word countin siltä sectionilta jos ei testi niin sitten näytä vain normaali section
\newcommand{\tsection}[2]{\istest{#1{#2\\} \wordcount \\ \medskip }{#1*{#2}}}


\newcommand{\pagesection}[1]{\istest{\section{#1}  }{\section*{#1}}}
\newcommand{\pagesubsection}[1]{\istest{\subsection{#1}  }{\subsection*{#1}}}
\newcommand{\oppagesubsection}[1]{\istest{\subsection{#1}  }{\subsection{#1}}}


% adds page with everything
\newcommand{\addPage}[2]{
    \def\filename{#1}
    \pagesection{#2}
    \istest{ \filewcount \medskip }{}
    \input{#1}
}



% OPPARI SPESIFIC

\newcommand{\addPageOp}[2]{
    \def\filename{#1}
    \oppagesubsection{#2}
    \istest{ \filewcount \medskip }{}
    \input{#1}
}

%jos pitää päivittää viitattu sanaa myöhemmin vaikka lisätä . tai : tai jotain
\def\viitattu{viitattu}

%lab citation style
\newcommand{\labcite}[1]{\setcitestyle{aysep={},open={(},close={.)}}\citep{#1}{}}

%lab citation style
\newcommand{\labciteend}[1]{\setcitestyle{aysep={},open={(},close={).}}\citep{#1}{}}

\newcommand{\labimgcite}[1]{\setcitestyle{aysep={},open={(},close={)}}\citep{#1}{}}

\newcommand{\citemissing}{\textbf{( SRC ? )}}

%\newcommand{\hurl}[1]{\href{#1}{{\underline{\textcolor{blue}{#1}}}}}
\newcommand{\hurl}[1]{\textcolor{blue}{\url{#1}}}


% counter kuville
\newcounter{imgCounter}
\setcounter{imgCounter}{0}

\newcommand{\getImgCount}{\addtocounter{imgCounter}{1}\theimgCounter}

\newcommand{\nextImageCount}{\the\numexpr \theimgCounter + 1 }
\newcommand{\nextnextImageCount}{\the\numexpr \theimgCounter + 2 }
\newcommand{\prevImageCount}{\the\numexpr \theimgCounter - 1 }

% counter kaavioille
\newcounter{chartCounter}
\setcounter{chartCounter}{0}

\newcommand{\getChartCount}{
\addtocounter{chartCounter}{1}
\thechartCounter
}

