
Kuudennella viikolla sain tykösi tehdä käyttäjien antamasta datasta raportti.
käyttäjät antavat sivuston kautta tietoja treeneistään ja muista seurattavista jne. sitten jossain vaiheessa tehdään raportti käyttäjien toiminnasta.
tämä raportti pitäisi nyt implementoida
%jotain lisää metatekstiä tästä  tai tästä viikosta


\subsection*{meteor}

%suomenna ja kirjota paremmin. meteor on osittain avoimeen lähdekoodiin perustuve framework
% paitsi mongodb jonka meteor paketoi mukaan
% pitää mainita miksi mongo sanotaan tässä ennen kun se sanotaan
% eli meteor tuo mukanaan mongon by default, joka on ainoa ei avoimeen lähdekoodiin perustuva
Meteor on, mongoDB tietokantaa lukuunottamatta, avoimeen lähdekoodiin perustuva "full stack" web-framework. Meteorilla on integraatio moneen yleiseen frontend frameworkkiin kuten vue, react, svelte tai angular.
%jotenkin paremmin liima aijempaan lauseeseen
meteoria voi käyttää monella eri frontend frameworkilla.
% suomenna
Meteor käyttää publisher-subscriber mallia clientin ja serverin välillä, joka antaa mahdollisuuden fronend päivitykselle kun tietokannassa tapahtuu muutos.
%jotain lisää ja meteorista ehkä?

%kuitenkin samaa tulee olemaan itse opparissa niin  emt pitääkö kirjoittaa tähän enemmän textiä

% tai toinen kappale

\medskip


\subsection*{subscriber publisher}

% täydellä lauseella ei aloiteta näin
Meteorin ratkaisu databasejen kanssa tekemiselle ei tarvitse erillistä sockettia tai uudelleen kutsua palvelimen ja käyttäjän välillä kun data päivittyy
% esim react ei varmaan ole hyvä voidaan laittaa vain että client saa tiedon
esim react komponentit voivat päivittyä automaattisesti jos ne ovet kuuntelemassa samaa tietolähdettä.
tekee devaamistesta helppoa.

miten projekti hyötyää tästä. ei paljoa sillä meidän ei tarvitse käyttää tätä
% aina kun data päivittyy servulla niin tarvittavat komponentit päivittyy automaattisesti
% yksinkertaiset db queryt ei tarvitse erillistä methodia vaan toimii vain publisher subscribe mallilla meteorissa
% vähentää methodien määrää ja kasvattaa koodin luettavuuttaa
\medskip



\subsection*{methodit}
mikä on method ja miski se on olemassa

% meteorin ratkaisu ei ole hyvä 
Meteor methodit on meteorin ratkaisu client-server rajapinnalle. \\
%
meteor ei käytä RESTful rajapintaa vaan meteor methodeja.
%  jotenkin vähän paremmin mutta asia on hyvä. pitää sanoa että ne suoriutuu palvelimella.
methodit suoritetaan ja määritetään itse palvelin puolella mutta niitä voi kutsua clientistä tai palvelimesta.
Methodeista voi helposti katsoa onko kutsuja kirjautunut sisään ja mitä oikeuksia käyttäjällä on. tämä tuo lisä turvallisuutta "out of the box", toisin kun RESTful apilla jossa se pitäisi itse implementoida
\medskip

miten methodeja käytetään projektissa\\
%sano suoraan Projektissa käytettään methodeja kun a b c ei kun d e f
methodeja ei käytetä kun halutaan tietokannasta, jotain yksinkertaista dataa, sillä publisher-subscriber malli hoitaa tämän
vaan methodeja käytetään silloin kun pitää tehdä jotain jota voi tehdä vain palvelimella. esim lähettää sähköposteja+*? ja tiedostojen lataaminen pilveen 
\medskip



\subsection*{Raportin luonti}

%selostus raportin specista ja mitä sen pitää tehdä
raportti jossa on. käyttäjän nimi, monta treenikertaa heillä on, seurattavan jutun ensimmäinen va viimeinen arvo.
raportti pitää pystyä avaamaan excelissä.
helposti tehtävä. jotain lisää
\medskip


% selostus miten raportti toimii käytännössä. kirjoita kirjakielellä
client kutsuu methodia ja antaa sille options objektin jossa kerrotaan mitä tietoa dokumenttiin halutaan. method varmistaa että käyttäjällä on tarvittavat oikeudet suorittaa methodi.
palvelin käy läpi asetukset mitä raporttiin halutaan ja lähettää datan takaisin clientille.
client muodostaa datasta tiedoston (csv). tämä on tehty siksi että tiedoston tekeminen olisi yleensä blocking ja ei haluta että palvelin stallaa silloin kun joku tekee jotain isoa dokumenttia.
kun client tekee datan muuntamisesta tiedostoon työn niin ei palvelin ota ylimääräistä stressiä.
kun tiedosto on tehty käyttäjä voi ladata sen.
\medskip


\subsection*{yhteenveto}
submary mitä tehtiin

meteor on semi ok
raportin luonti saatiin tehtyä ja se toimii miten haluttiin

