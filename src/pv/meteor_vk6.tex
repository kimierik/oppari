
Kuudennella viikolla sain tykösi tehdä käyttäjien antamasta datasta raportti.
käyttäjät antavat sivuston kautta tietoja treeneistään ja muista seurattavista jne. sitten jossain vaiheessa tehdään raportti käyttäjien toiminnasta.
tämä raportti pitäisi nyt implementoida
%jotain lisää metatekstiä tästä  tai tästä viikosta


\subsection*{meteor}

%suomenna ja kirjota paremmin. meteor on osittain avoimeen lähdekoodiin perustuve framework
% paitsi mongodb jonka meteor paketoi mukaan
% pitää mainita miksi mongo sanotaan tässä ennen kun se sanotaan
% eli meteor tuo mukanaan mongon by default, joka on ainoa ei avoimeen lähdekoodiin perustuva
Meteor on, mongoDB tietokantaa lukuunottamatta, avoimeen lähdekoodiin perustuva "full stack" web-framework. Meteorilla on integraatio moneen yleiseen frontend frameworkkiin kuten vue, react, svelte tai angular.
%jotenkin paremmin liima aijempaan lauseeseen
meteoria voi käyttää monella eri frontend frameworkilla.
% suomenna
Meteor käyttää publisher-subscriber mallia clientin ja serverin välillä, joka antaa mahdollisuuden fronend päivitykselle kun tietokannassa tapahtuu muutos.
%jotain lisää ja meteorista ehkä?

%kuitenkin samaa tulee olemaan itse opparissa niin  emt pitääkö kirjoittaa tähän enemmän textiä

% tai toinen kappale

\medskip




%https://www.leviathansecurity.com/blog/websockets-and-meteor-a-penetration-testers-guide-to-meteor

% ddp distibuted data protocol


\subsection*{subscriber publisher}

%fine ? emt onko yleinen hyvä alku voisi kirjoittaa jotenkin toisin
Yleinen ongelma Web-sovelluksissa on käyttöliittymän päivitys kun palvelimella tai tietokannassa tapahtuu muutoksia. 
% tavallisissa nettisivuissa on tosi huono. vain getrequest.
Tavallisissa nettisivuissa tehtäisiin vain get request palvelimelle ja palvelin antaisi pyydetyn datan. Client ei tietäisi jos data on vanhaa tai jos se on päivittynyt palvelimella.
%Yleisesti datan saaminen palvelimenlta/tietokannalta käyttöliittymälle, käytetään Get requesteja
normaalisti jos halutaan dataa käyttöliittymälle palvelimelta/tietokannasta, käytetään GET requesteja. mutta näin data tulee ainoastaan kun client kysyy sitä, eikä silloin kun data muuttuu
Jos clientin yhdistää palvelimenlle web-socketilla, palvelin voi ilmoittaa clientille jos data muuttuu tietokannassa.
% toinen ratkaisu
%toinen ratkaisu on yhdistää client ja palvelin web-socetilla, jolloin palvelin voi ilmoittaa kun data muuttuu.
\medskip

%meteor käyttää websocketteja sub-publisher jutussa. 
%Meteorin ratkaisu tähän on käyttää "publisher-subscriber" mallia.
Meteor käyttää websocketteja sen publisher-subscriber subscriber mallin toteutuksessa.
%
Palvelin voi julkaista tietokannan osia meteor.publish-funktiolla, jos asiakas tilaa tämän julkaisun, se saa ilmoituksen aina, kun tiedot, jotka asiakas on tilannut, muuttuvat.
Meteor voi tällöin päivittää käyttöliittymän osia, jotka ovat riippuvaisia kyseisistä tiedoista. 
% kuvia vaikka publish ja subscribe kutsuista
%
% esimerkki kuva publishista mongodb käytössä
%esimerkki client subscirbestä 
%https://docs.meteor.com/api/pubsub#Meteor-subscribe
%tästä tehdään esimerkki koodi ja siitä screenshot
%
%jotenkin paremmin selittää 
Kulissien takana meteor käyttää websocketteja tämän toteuttamiseen. 
\medskip

% esim react ei varmaan ole hyvä voidaan laittaa vain että client saa tiedon
%esim react komponentit voivat päivittyä automaattisesti jos ne ovet kuuntelemassa samaa tietolähdettä.

Esimerkiksi jos Meteoria käytetään Reactin kanssa, se voi päivittää tarvittavat React komponentit tietojen muututtua tietokannassa.

%miten projekti hyötyää tästä. ei paljoa sillä meidän ei tarvitse käyttää tätä
% aina kun data päivittyy servulla niin tarvittavat komponentit päivittyy automaattisesti
% yksinkertaiset db queryt ei tarvitse erillistä methodia vaan toimii vain publisher subscribe mallilla meteorissa
% vähentää methodien määrää ja kasvattaa koodin luettavuuttaa


\medskip



\subsection*{Methodit}

Meteor ei käytä RESTful rajapintaa palvelimen ja clientin välillä, vain Meteorin methodeja.

%  jotenkin vähän paremmin mutta asia on hyvä. pitää sanoa että ne suoriutuu palvelimella.
%metodi on vain funktio
Methodit suoritetaan ja määritetään itse palvelin puolella, mutta niitä voi kutsua clientistä tai palvelimesta.
% Methodid määriteellään palvelimen puolella funktiona. funktio saa this kontekstin mukana kirjautuneen käyttäjän userIdn, jos kutsuja on kirjautunut sisään.
% this. method funktiossa
%
% method funktio saa this kontekstissa funktio saa esim tiedon jos kutsuja on kirjautunut sisään. ja sen useridn
% pitää paremmin selittää restful apin epäturvallisuudesta  https://guide.meteor.com/methods
Methodeista voi helposti katsoa onko kutsuja kirjautunut sisään ja mitä oikeuksia käyttäjällä on. tämä tuo lisä turvallisuutta "out of the box", toisin kun RESTful apilla, jossa turvallisuus pitäisi itse implementoida.
\medskip

% kuva methodi määritelmästä ja seloitus mitä tapahtuu

%sano suoraan Projektissa käytettään methodeja kun a b c ei kun d e f
%emt tarviiko ees selittää
methodeja ei käytetä kun halutaan tietokannasta, jotain yksinkertaista dataa, sillä publisher-subscriber malli hoitaa tämän
vaan methodeja käytetään silloin kun pitää tehdä jotain jota voi tehdä vain palvelimella. esim lähettää sähköposteja+*? ja tiedostojen lataaminen pilveen 
\medskip



\subsection*{Raportin luonti}

%selostus raportin specista ja mitä sen pitää tehdä
raportti jossa on. käyttäjän nimi, monta treenikertaa heillä on, seurattavan jutun ensimmäinen va viimeinen arvo.
raportti pitää pystyä avaamaan excelissä.
helposti tehtävä. jotain lisää
\medskip


% selostus miten raportti toimii käytännössä. kirjoita kirjakielellä
client kutsuu methodia ja antaa sille options objektin jossa kerrotaan mitä tietoa dokumenttiin halutaan. method varmistaa että käyttäjällä on tarvittavat oikeudet suorittaa methodi.
palvelin käy läpi asetukset mitä raporttiin halutaan ja lähettää datan takaisin clientille.
client muodostaa datasta tiedoston (csv). tämä on tehty siksi että tiedoston tekeminen olisi yleensä blocking ja ei haluta että palvelin stallaa silloin kun joku tekee jotain isoa dokumenttia.
kun client tekee datan muuntamisesta tiedostoon työn niin ei palvelin ota ylimääräistä stressiä.
kun tiedosto on tehty käyttäjä voi ladata sen.
\medskip


\subsection*{yhteenveto}
submary mitä tehtiin

meteor on semi ok
raportin luonti saatiin tehtyä ja se toimii miten haluttiin

